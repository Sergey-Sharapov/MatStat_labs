\documentclass[a4]{article}
\pagestyle{myheadings}

%%%%%%%%%%%%%%%%%%%
% Packages/Macros %
%%%%%%%%%%%%%%%%%%%
\usepackage{mathrsfs}


\usepackage{fancyhdr}
\pagestyle{fancy}
\lhead{}
\chead{}
\rhead{}
\lfoot{}
\cfoot{} 
\rfoot{\normalsize\thepage}
\renewcommand{\headrulewidth}{0pt}
\renewcommand{\footrulewidth}{0pt}
\newcommand{\RomanNumeralCaps}[1]
    {\MakeUppercase{\romannumeral #1}}

\usepackage{amssymb,latexsym}  % Standard packages
\usepackage[utf8]{inputenc}
\usepackage[russian]{babel}
\usepackage{MnSymbol}
\usepackage{mathrsfs}
\usepackage{amsmath,amsthm}
\usepackage{indentfirst}
\usepackage{graphicx}%,vmargin}
\usepackage{graphicx}
\graphicspath{{pictures/}} 
\usepackage{verbatim}
\usepackage{color}
\usepackage{color,colortbl}
\usepackage[nottoc,numbib]{tocbibind}
\usepackage{float}
\usepackage{multirow}
\usepackage{hhline}

\usepackage{listings}
\definecolor{codegreen}{rgb}{0,0.6,0}
\definecolor{codegray}{rgb}{1,1,1}
\definecolor{codepurple}{rgb}{0.58,0,0.82}
\definecolor{backcolour}{rgb}{0.95,0.95,0.92}
 
\lstdefinestyle{mystyle}{
    backgroundcolor=\color{backcolour},   
    commentstyle=\color{codegreen},
    keywordstyle=\color{magenta},
    numberstyle=\tiny\color{codegray},
    stringstyle=\color{codepurple},
    basicstyle=\footnotesize,
    breakatwhitespace=false,         
    breaklines=true,                 
    captionpos=b,                    
    keepspaces=true,                 
    numbers=left,                    
    numbersep=5pt,                  
    showspaces=false,                
    showstringspaces=false,
    showtabs=false,                  
    tabsize=2
}
 
\lstset{style=mystyle}

\usepackage{url}
\urldef\myurl\url{foo%.com}
\def\UrlBreaks{\do\/\do-}
\usepackage{breakurl}
\Urlmuskip=0mu plus 1mu



\DeclareGraphicsExtensions{.pdf,.png,.jpg}% -- настройка картинок

\usepackage{epigraph} %%% to make inspirational quotes.
\usepackage[all]{xy} %for XyPic'a
\usepackage{color} 
\usepackage{amscd} %для коммутативных диграмм
%\usepackage[colorlinks,urlcolor=red]{hyperref}

%\renewcommand{\baselinestretch}{1.5}
%\sloppy
%\usepackage{listings}
%\lstset{numbers=left}
%\setmarginsrb{2cm}{1.5cm}{1cm}{1.5cm}{0pt}{0mm}{0pt}{13mm}


\newtheorem{Lemma}{Лемма}[section]
\newtheorem{Proposition}{Предложение}[section]
\newtheorem{Theorem}{Теорема}[section]
\newtheorem{Corollary}{Следствие}[section]
\newtheorem{Remark}{Замечание}[section]
\newtheorem{Definition}{Определение}[section]
\newtheorem{Designations}{Обозначение}[section]




%%%%%%%%%%%%%%%%%%%%%%% 
%Подготовка оглавления% 
%%%%%%%%%%%%%%%%%%%%%%% 
\usepackage[titles]{tocloft}
\renewcommand{\cftdotsep}{2} %частота точек
\renewcommand\cftsecleader{\cftdotfill{\cftdotsep}}
\renewcommand{\cfttoctitlefont}{\hspace{0.38\textwidth} \LARGE\bfseries} 
\renewcommand{\cftsecaftersnum}{.}
\renewcommand{\cftsubsecaftersnum}{.}
\renewcommand{\cftbeforetoctitleskip}{-1em} 
\renewcommand{\cftaftertoctitle}{\mbox{}\hfill \\ \mbox{}\hfill{\footnotesize Стр.}\vspace{-0.5em}} 
%\renewcommand{\cftchapfont}{\normalsize\bfseries \MakeUppercase{\chaptername} } 
%\renewcommand{\cftsecfont}{\hspace{1pt}} 
\renewcommand{\cftsubsecfont}{\hspace{1pt}} 
%\renewcommand{\cftbeforechapskip}{1em} 
\renewcommand{\cftparskip}{3mm} %определяет величину отступа в оглавлении
\setcounter{tocdepth}{5} 
\renewcommand{\listoffigures}{\begingroup %добавляем номер в список иллюстраций
\tocsection
\tocfile{\listfigurename}{lof}
\endgroup}
\renewcommand{\listoftables}{\begingroup %добавляем номер в список иллюстраций
\tocsection
\tocfile{\listtablename}{lot}
\endgroup}


%\renewcommand{\thelikesection}{(\roman{likesection})}
%%%%%%%%%%%
% Margins %
%%%%%%%%%%%
\addtolength{\textwidth}{0.7in}
\textheight=630pt
\addtolength{\evensidemargin}{-0.4in}
\addtolength{\oddsidemargin}{-0.4in}
\addtolength{\topmargin}{-0.4in}

%%%%%%%%%%%%%%%%%%%%%%%%%%%%%%%%%%%
%%%%%%Переопределение chapter%%%%%% 
%%%%%%%%%%%%%%%%%%%%%%%%%%%%%%%%%%%
\newcommand{\empline}{\mbox{}\newline} 
\newcommand{\likechapterheading}[1]{ 
\begin{center} 
\textbf{\MakeUppercase{#1}} 
\end{center} 
\empline} 

%%%%%%%Запиливание переопределённого chapter в оглавление%%%%%% 
\makeatletter 
\renewcommand{\@dotsep}{2} 
\newcommand{\l@likechapter}[2]{{\bfseries\@dottedtocline{0}{0pt}{0pt}{#1}{#2}}} 
\makeatother 
\newcommand{\likechapter}[1]{ 
\likechapterheading{#1} 
\addcontentsline{toc}{likechapter}{\MakeUppercase{#1}}} 




\usepackage{xcolor}
\usepackage{hyperref}
\definecolor{linkcolor}{HTML}{000000} % цвет ссылок
\definecolor{urlcolor}{HTML}{3643FF} % цвет гиперссылок
 
\hypersetup{pdfstartview=FitH,  linkcolor=linkcolor,urlcolor=urlcolor, colorlinks=true}

%%%%%%%%%%%%
% Document %
%%%%%%%%%%%%

%%%%%%%%%%%%%%%%%%%%%%%%%%%%%
%%%%%%главы -- section*%%%%%%
%%%%section -- subsection%%%%
%subsection -- subsubsection%
%%%%%%%%%%%%%%%%%%%%%%%%%%%%%
\def \newstr {\medskip \par \noindent} 



\begin{document}
\newcolumntype{g}{>{\columncolor{codegray}}c}



\def\contentsname{\LARGE{Содержание}}
\thispagestyle{empty}
\begin{center} 
	\vspace{2cm} 
	{\Large \sc Санкт-Петербургский Политехнический Университет}\\
	\vspace{2mm}
	{\Large\sc им. Петра Великого}\\
	\vspace{1cm}
	{\large \sc Институт прикладной математики и механики\\ 
		\vspace{0.5mm}
		\textsc{}}\\ 
	\vspace{0.5mm}
	{\large\sc Кафедра $"$Прикладная математика$"$}\\
	\vspace{15mm}
	
	
	{\sc \textbf{Отчёт\\
			Курсовая работа\\
			по дисциплине\\
			"Математическая статистика"}
		\vspace{6mm}
		
	}
	\vspace*{2mm}
	
	
	\begin{flushleft}
		\vspace{4cm}
		\sc Выполнил студент:\\
		\sc Шарапов Сергей Андреевич\\
		\sc группа: 3630102/70401\\
		\vspace{1cm}
		\sc Проверил:\\
		\sc к.ф-м.н., доцент\\
		\sc Баженов Александр Николаевич
		\vspace{20mm}
	\end{flushleft}
\end{center} 
\begin{center}
	\vfill {\large\textsc{Санкт-Петербург}}\\ 
	2020 год
\end{center}

\newpage
\pagestyle{plain}



\newpage
\tableofcontents{}
\newpage
\listoftables{}
\newpage

\section{Постановка задачи}
Провести исследования о распознавании распределения Лапласа как нормального. Генерировать выборки по закону распределения Лапласа, методом максимального правдоподобия оценивать параметры нормального распределения($\mu$ и $\sigma$), предполагая, что полученная выборка может подчиняться нормальному закону. В итоге требуется использовать критерий согласия Пирсона для определения сходства распределений Лапласа и нормального распределения. При заданных параметрах изучить зависимоcть от мощности выборки $n = 50, 100, 200, 1000$. При мощности выборки $n = 100$ варьировать параметр распределения Лапласа: $\alpha = 0.5, 1, 2, 4$. В качестве уровня значимости взять $\alpha=0,05.$

\section{Теория}
\subsection{Метод максимального правдоподобия}
Метод максимального правдоподобия $\--$ метод оценивания неизвестного параметра путём максимимзации функции правдоподобия.
\begin{equation}
    \overset{\wedge}{\theta}_{\text{МП}}=argmax \mathbf{L}(x_1,x_2,\ldots,x_n,\theta)
\end{equation}

Где $\mathbf{L}$ это функция правдоподобия, которая представляет собой совместную плотность вероятности независимых случайных величин $X_1,x_2,\ldots,x_n$ и является функцией неизвестного параметра $\theta$
\begin{equation}
    \mathbf{L} = f(x_1,\theta)\cdot f(x_2,\theta)\cdot\cdots\cdot f(x_n,\theta)
\end{equation}
Оценкой максимального правдоподобия будем называть такое значение $\overset{\wedge}{\theta}_{\text{МП}}$ из множества допустимых значений параметра $\theta,$ для которого функция правдоподобия принимает максимальное значение при заданных $x_1,x_2,\ldots,x_n.$

Тогда при оценивании математического ожидания $m$ и дисперсии $\sigma^2$ нормального распределения $N(m,\sigma)$ получим:
\begin{equation}
    \ln(\mathbf{L})=-\frac{n}{2}\ln(2\pi)-\frac{n}{2}\ln\left(\sigma^2\right)-\frac{1}{2\sigma^2}\sum\limits_{i=1}^n(x_i-m)^2
\end{equation}

Отсюда находятся выражения для оценок $m$ и $\sigma^2$:
\begin{equation}
\begin{cases}
&  m= \bar{x}\\ 
&  \sigma^{2} = s^{2}
\end{cases}
\end{equation}

\subsection{Критерий согласия Пирсона}
Разобьём генеральную совокупность на $k$ неперсекающихся подмножеств $\Delta_1, \Delta_2,\ldots, \Delta_k,\;\Delta_i = (a_i,a_{i+1}],$ $p_i = P(X\in\Delta_i),\;i=1,2,\ldots,k\; \--$ вероятность того, что точка попала в $i$ый промежуток.

Так как генеральная совокупность это $\mathbb{R},$ то крайние промежутки будут бесконечными: $\Delta_1=(-\infty,a_1],\;\Delta_k=(a_k,\infty),\;p_i = F(a_i)-F(a_{i-1})$

$n_i\;\--$ частота попадания выборочных элементов в $\Delta_i,\;i=1,2,\ldots,k.$

В случае справедливости гипотезы $H_0$ относительно частоты $\frac{n_i}{n}$ при больших $n$ должны быть близки к $p_i,$ значит в качестве меры имеет смысл взять: 
\begin{equation}
    Z = \sum\limits_{i=1}^k\frac{n}{p_i}\left(\frac{n_i}{n}-p_i\right)^2
\end{equation}
Тогда
\begin{equation}
    \chi^2_B=\sum\limits_{i=1}^k\frac{n}{p_i}\left(\frac{n_i}{n}-p_i\right)^2=\sum\limits_{i=1}^k\frac{(n_i-np_i)^2}{np_i}
\end{equation}
Для выполнения гипотезы $H_0$ должны выполняться следующие условия:
\begin{equation}
    \chi_B^2 < \chi_{1-\alpha}^2(k-1)
\end{equation}
где $\chi_{1-\alpha}^2(k-1)\;\--$ квантиль распределения $\chi^2$ с $k-1$ степенями свободы порядка $1-\alpha,$ где $\alpha$ заданный уровень значимости.
\section{Реализация}
Работа была выполнена на языке $Python 3.7$
Для генерации выборок и обработки функции распределения использовалась библиотека $scipy.stats$.

\section{Результаты}


\subsection{Исследование зависимости распознования распределения Лапласа как нормального от величины выборки}
Дано распределение Лапласа:
\begin{equation}\label{eqn:laplace}
L\left( x,0,\frac{1}{\sqrt{2}}\right) = \frac{1}{2\sqrt{2}}e^{-\frac{\vert x\vert}{\sqrt{2}}}
\end{equation}
Размер выборки $n = 50$:
  \begin{equation}
  \begin{split}
  &\overset{\wedge}{m}_{\text{МП}} =-0.0692\\
  &  \overset{\wedge}{\sigma}^2_{\text{МП}} = 1.0152
  \end{split}
  \end{equation}

\begin{table}[H]
	\caption{Таблица вычислений $\chi^2, n = 50$}
	\label{tab:my_label1}
	\begin{center}
		\vspace{5mm}
		\begin{tabular}{|c|c|c|c|c|}
			\hline
			i & $\Delta_i$ & $n_i$ & $p_i$ & $\frac{(n_i-np_i)^2}{np_i}$\\ \hline
			0&  ($-\infty ,-1.0$]&  8&  0.1796&  0.1070\\ \hline
			1&  ($-1.0,-0.5$]&  7&  0.1561&  0.0826\\ \hline
			2&  ($-0.5,0.0$]&  15&  0.1915&  3.0717\\ \hline
			3&  ($0.0,0.5$]&  6&  0.1853&  1.1513\\ \hline
			4&  ($0.5,1.0$]&  7&  0.1414&  0.0007\\ \hline
			5&  ($ 1.0,\infty$)&  7&  0.1461&  0.0128\\ \hline
		\end{tabular}
	\end{center}
\end{table}

$$\chi_B^2 = 4.4261$$ 

Размер выборки $n = 100$:
  \begin{equation}
  \begin{split}
  &\overset{\wedge}{m}_{\text{МП}} =0.0746\\
  &  \overset{\wedge}{\sigma}^2_{\text{МП}} = 0.8623
  \end{split}
  \end{equation}

\begin{table}[H]
	\caption{Таблица вычислений $\chi^2$, n = 100}
	\label{tab:my_label1}
	\begin{center}
		\vspace{5mm}
		\begin{tabular}{|c|c|c|c|c|}
			\hline
			i & $\Delta_i$ & $n_i$ & $p_i$ & $\frac{(n_i-np_i)^2}{np_i}$\\ \hline
			0&  ($-\infty ,-1.0$]&  9&  0.1063&  0.2510\\ \hline
			1&  ($-1.0,-0.5$]&  9&  0.1462&  2.1633\\ \hline
			2&  ($-0.5,0.0$]&  26&  0.2129&  1.0398\\ \hline
			3&  ($0.0,0.5$]&  33&  0.2236&  5.0643\\ \hline
			4&  ($0.5,1.0$]&  9&  0.1693&  3.7141\\ \hline
			5&  ($ 1.0,\infty$)&  14&  0.1416&  0.0018\\ \hline
		\end{tabular}
	\end{center}
\end{table}

$$\chi_B^2 = 12.2342$$ 

Размер выборки $n = 200$:
  \begin{equation}
  \begin{split}
  &\overset{\wedge}{m}_{\text{МП}} =-0.0052\\
  &  \overset{\wedge}{\sigma}^2_{\text{МП}} = 1.0047
  \end{split}
  \end{equation}

\begin{table}[H]
	\caption{Таблица вычислений $\chi^2$, n = 200}
	\label{tab:my_label1}
	\begin{center}
		\vspace{5mm}
		\begin{tabular}{|c|c|c|c|c|}
			\hline
			i & $\Delta_i$ & $n_i$ & $p_i$ & $\frac{(n_i-np_i)^2}{np_i}$\\ \hline
			0&  ($-\infty ,-1.0$]&  26&  0.1611&  1.1979\\ \hline
			1&  ($-1.0,-0.75$]&  10&  0.0682&  0.9715\\ \hline
			2&  ($-0.75,-0.5$]&  10&  0.0819&  2.4898\\ \hline
			3&  ($-0.5,-0.25$]&  23&  0.0926&  1.0879\\ \hline
			4&  ($-0.25,0.0$]&  30&  0.0983&  5.4340\\ \hline
			5&  ($0.0,0.25$]&  33&  0.0982&  9.0920\\ \hline
			6&  ($0.25,0.5$]&  16&  0.0922&  0.3231\\ \hline
			7&  ($0.5,0.75$]&  17&  0.0814&  0.0316\\ \hline
			8&  ($0.75,1.0$]&  10&  0.0676&  0.9155\\ \hline
			9&  ($ 1.0,\infty$)&  25&  0.1585&  1.4183\\ \hline
		\end{tabular}
	\end{center}
\end{table}

$$\chi_B^2 = 22.9616$$ 

Размер выборки $n = 1000$:
  \begin{equation}
  \begin{split}
  &\overset{\wedge}{m}_{\text{МП}} =-0.0210\\
  &  \overset{\wedge}{\sigma}^2_{\text{МП}} = 0.9983
  \end{split}
  \end{equation}

\begin{table}[H]
	\caption{Таблица вычислений $\chi^2$, n = 1000}
	\label{tab:my_label1}
	\begin{center}
		\vspace{5mm}
		\begin{tabular}{|c|c|c|c|c|}
			\hline
			i & $\Delta_i$ & $n_i$ & $p_i$ & $\frac{(n_i-np_i)^2}{np_i}$\\ \hline
			0&  ($-\infty ,-1.0$]&  129&  0.1634&  7.2424\\ \hline
			1&  ($-1.0,-0.8571$]&  28&  0.0378&  2.5226\\ \hline
			2&  ($-0.8571,-0.7143$]&  35&  0.0426&  1.3424\\ \hline
			3&  ($-0.7143,-0.5714$]&  39&  0.0470&  1.3606\\ \hline
			4&  ($-0.5714,-0.4286$]&  45&  0.0508&  0.6724\\ \hline
			5&  ($-0.4286,-0.2857$]&  66&  0.0539&  2.7162\\ \hline
			6&  ($-0.2857,-0.1429$]&  74&  0.0560&  5.8002\\ \hline
			7&  ($-0.1429,0.0$]&  96&  0.0570&  26.7487\\ \hline
			8&  ($0.0,0.1429$]&  99&  0.0568&  31.3657\\ \hline
			9&  ($0.1429,0.2857$]&  69&  0.0555&  3.2962\\ \hline
			10&  ($0.2857,0.4286$]&  55&  0.0531&  0.0684\\ \hline
			11&  ($0.4286,0.5714$]&  41&  0.0498&  1.5507\\ \hline
			12&  ($0.5714,0.7143$]&  47&  0.0457&  0.0347\\ \hline
			13&  ($0.7143,0.8571$]&  29&  0.0412&  3.5985\\ \hline
			14&  ($0.8571,1.0$]&  30&  0.0363&  1.0966\\ \hline
			15&  ($ 1.0,\infty$)&  118&  0.1532&  8.0947\\ \hline
		\end{tabular}
	\end{center}
\end{table}

$$\chi_B^2 = 97.5109$$ 
\subsection{Исследование зависимости распознования распределения Лапласа как нормального от параметра $\alpha$ распределения}
Дано распределение Лапласа:
\begin{equation}\label{eqn:laplace}
L\left( x,0,\alpha\right) = \frac{\alpha}{2}e^{-\alpha\vert x\vert}
\end{equation}
Размер выборки $n = 100$. \\
$\alpha= \frac{1}{2}$:
  \begin{equation}
  \begin{split}
  &\overset{\wedge}{m}_{\text{МП}} =0.1507\\
  &  \overset{\wedge}{\sigma}^2_{\text{МП}} = 0.5960
  \end{split}
  \end{equation}

\begin{table}[H]
	\caption{Таблица вычислений $\chi^2, \alpha= \frac{1}{2}$}
	\label{tab:my_label1}
	\begin{center}
		\vspace{5mm}
		\begin{tabular}{|c|c|c|c|c|}
			\hline
			i & $\Delta_i$ & $n_i$ & $p_i$ & $\frac{(n_i-np_i)^2}{np_i}$\\ \hline
0&  ($-\infty ,-1.0$]&  3&  0.0268&  0.0391\\ \hline
1&  ($-1.0,-0.5$]&  4&  0.1107&  4.5154\\ \hline
2&  ($-0.5,0.0$]&  35&  0.2627&  2.9011\\ \hline
3&  ($0.0,0.5$]&  34&  0.3209&  0.1140\\ \hline
4&  ($0.5,1.0$]&  17&  0.2019&  0.5026\\ \hline
5&  ($ 1.0,\infty$)&  7&  0.0771&  0.0655\\ \hline
		\end{tabular}
	\end{center}
\end{table}

$$\chi_B^2 = 8.1377$$ 

$\alpha= 1$:
  \begin{equation}
  \begin{split}
  &\overset{\wedge}{m}_{\text{МП}} =0.0157\\
  &  \overset{\wedge}{\sigma}^2_{\text{МП}} = 1.2034
  \end{split}
  \end{equation}

\begin{table}[H]
	\caption{Таблица вычислений $\chi^2, \alpha= 1$}
	\label{tab:my_label1}
	\begin{center}
		\vspace{5mm}
		\begin{tabular}{|c|c|c|c|c|}
			\hline
			i & $\Delta_i$ & $n_i$ & $p_i$ & $\frac{(n_i-np_i)^2}{np_i}$\\ \hline
0&  ($-\infty ,-1.0$]&  15&  0.1993&  1.2205\\ \hline
1&  ($-1.0,-0.5$]&  10&  0.1348&  0.8983\\ \hline
2&  ($-0.5,0.0$]&  23&  0.1607&  2.9928\\ \hline
3&  ($0.0,0.5$]&  24&  0.1615&  3.8129\\ \hline
4&  ($0.5,1.0$]&  9&  0.1370&  1.6114\\ \hline
5&  ($ 1.0,\infty$)&  19&  0.2067&  0.1352\\ \hline
		\end{tabular}
	\end{center}
\end{table}

$$\chi_B^2 =10.6710$$ 

$\alpha= 2$:
  \begin{equation}
  \begin{split}
  &\overset{\wedge}{m}_{\text{МП}} = 0.2995\\
  &  \overset{\wedge}{\sigma}^2_{\text{МП}} = 2.5786
  \end{split}
  \end{equation}

\begin{table}[H]
	\caption{Таблица вычислений $\chi^2, \alpha= 2$}
	\label{tab:my_label1}
	\begin{center}
		\vspace{5mm}
		\begin{tabular}{|c|c|c|c|c|}
			\hline
			i & $\Delta_i$ & $n_i$ & $p_i$ & $\frac{(n_i-np_i)^2}{np_i}$\\ \hline
0&  ($-\infty ,-1.0$]&  26&  0.3071&  0.7235\\ \hline
1&  ($-1.0,-0.5$]&  4&  0.0711&  1.3613\\ \hline
2&  ($-0.5,0.0$]&  6&  0.0755&  0.3185\\ \hline
3&  ($0.0,0.5$]&  17&  0.0772&  11.1471\\ \hline
4&  ($0.5,1.0$]&  11&  0.0761&  1.5135\\ \hline
5&  ($ 1.0,\infty$)&  36&  0.3929&  0.2763\\ \hline
		\end{tabular}
	\end{center}
\end{table}

$$\chi_B^2 =15.3401$$ 

$\alpha= 4$:
  \begin{equation}
  \begin{split}
  &\overset{\wedge}{m}_{\text{МП}} = -0.6772\\
  &  \overset{\wedge}{\sigma}^2_{\text{МП}} = 5.4524
  \end{split}
  \end{equation}

\begin{table}[H]
	\caption{Таблица вычислений $\chi^2, \alpha= 4$}
	\label{tab:my_label1}
	\begin{center}
		\vspace{5mm}
		\begin{tabular}{|c|c|c|c|c|}
			\hline
			i & $\Delta_i$ & $n_i$ & $p_i$ & $\frac{(n_i-np_i)^2}{np_i}$\\ \hline
0&  ($-\infty ,-1.0$]&  39&  0.4764&  1.5668\\ \hline
1&  ($-1.0,-0.5$]&  8&  0.0366&  5.1584\\ \hline
2&  ($-0.5,0.0$]&  7&  0.0365&  3.0855\\ \hline
3&  ($0.0,0.5$]&  8&  0.0360&  5.3594\\ \hline
4&  ($0.5,1.0$]&  4&  0.0353&  0.0614\\ \hline
5&  ($ 1.0,\infty$)&  34&  0.3792&  0.4050\\ \hline
		\end{tabular}
	\end{center}
\end{table}

$$\chi_B^2 =15.6365$$ \\
Табличные значения квантилей: $\chi^2_{0.95}(5) = 11.0705,  \chi^2_{0.95}(15) = 24.9958$.
\section{Выводы}
Для выборки $n = 50$ распределения Лапласа полученно значение критерия Пирсона: $\chi_B^2 = 4.4261 < \chi^2_{0.95}(5) = 11.0705$. Это означает, что из полученной выборки мы не можем опровергнуть гипотезу о нормальности исходного распределения на уровне значимости $\alpha = 0.05$. При данных параметрах и размере выборки распределение Лапласа распознается как нормальное, что не правда. При увеличении выборки до 100, 200 или 1000 элементов значение критерия Пирсона возрастает, превышая соответствующие значения $\chi^2_{1-\alpha}$, и уже позволяет отвергнуть потенциальную нулевую гипотезу о нормальном распределении. В результате увеличения выборки в интервалы попадает больше точек, что сильно влияет на получаемое значение критерия.\\
\par Исследуя распределения Лапласа с параметрами $\alpha$ = 1/2 и 1, получилось, что они сходны с соотвествующими нормальными распределениями. Однако, с увеличением $\alpha$ росло и $\chi_B^2$, поэтому при $\alpha$ = 2 и 4 значение критерия Пирсона стало больше $\chi^2_{0.95}(5) = 11.0705$. Получается с увеличением параметра масштаба распределения Лапласа, оно становиться менее распознаваемым как нормальное.

\section{Литература}

\href{https://physics.susu.ru/vorontsov/language/numpy.html}{Модуль numpy}\\

\href{https://matplotlib.org/}{Модуль matplotlib}\\

\href{https://www.scipy.org/}{Модуль scipy}\\

Шевляков Г. Л. Лекции по математической статистике, 2019.


\section{Приложения}

\href{https://github.com/Sergey-Sharapov/MatStat_labs/blob/main/course_project/main.py}{Код курсового проекта}

\end{document}
